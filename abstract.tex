\begin{abstract}

Nell'Ingegneria del Software e' importante avere a disposizione tool automatici che permettano di controllare le violazioni delle politiche nel codice, tuttavia essi richiedono di analizzare relazioni e dipendenze dei componenti di una code base di grandi dimensioni e che puo' essere scritta con vari linguaggi di programmazione. La maggior parte dei tool presenti nel mercato del software proprietario o libero fanno uso di ulteriori software di terze parti per ottenere il grafo delle dipendenze di un certo progetto e questa operazione, oltre ad essere molto onerosa quando si decida di appoggiarsi a Language Server o moduli simili, richiede di usare un software diverso per ogni linguaggio di programmazione da analizzare. Lo scopo di questa ricerca e' verificare sotto quali condizioni si possa rimuovere la language dependency della costruzione del grafo delle dipendenze usano due librerie: Tree Sitter e Stack Graph. Dopo avere analizzato le due tecnologie proposte, il modello introdotto ad hoc e la sua implementazione, vengono valutate l'efficacia e l'accuratezza di questa soluzione.

\end{abstract}
