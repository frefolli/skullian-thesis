\chapter{Introduzione}

\section{Caso di Studio}

Arcan e' un tool sviluppato dalla startup Arcan Tech in grado di analizzare un progetto e calcolare il relativo tecnical debt e gli eventuali architectural smells. Per farlo si serve di un Dependency Graph, che crea usando delle chiamate ai language server (o librerie affini) dei linguaggi usati nel progetto.
Questo processo tuttavia risulta lento a causa del fatto che generalmente queste tecnologie non sono pensate per assolvere solo a questo scopo e di conseguenza contengono degli strati piu' sofisticati che appesantiscono l'esecuzione degli stessi. Non esistono tecnologie alternative che siano production ready o che assolvano a questi scopi e che siano competitive rispetto alle performance.

\section{L'Idea}

Dato questo technological gap si vuole studiare la fattibilita' di un sistema che usi l'analisi statica lessicale per ricavare le stesse informazioni in maniera piu' efficiente.
Si vuole anche rendere il piu' possibile indipendente il metodo rispetto ai linguaggi di programmazione analizzati.
L'idea e' quella di combinare le librerie Tree Sitter e Stack Graph con una serie di algoritmi specifici del problema per ottenere un sistema funzionante ed efficiente.

\section{Soluzione}

Usando il DSL \textbf{Tree Sitter Stack Graph} si costruisce una grammatica (PL-specific, quindi una per ogni linguaggio di programmazione) che possa essere utilizata per costruire a runtime lo \textbf{Stack Graph} del progetto.
Quindi si combina la risoluzione simbolica dello Stack Graph con un algoritmo di esplorazione dello stesso per costruire il Dependency Graph (questo Pl-independent).
