\chapter{Conclusioni}

Questo capitolo contiene un elenco delle difficolt\`a riscontrate e le conclusioni dello studio.

\section{Problemi Riscontrati}

\paragraph{Problemi affrontati}

%\paragraph{Parallelizzazione}
Nell'ottica di scalare verticalmente le risorse impiegate dal prototipo si \`e posto il problema di ideare un processo che fosse parallelizzabile in modo sicuro. Ebbene la parte che si \`e sperimentalmente ricavata essere quella pi\`u impattante \`e la risoluzione delle reference: essa pu\`o essere parallelizzata senza problemi visto che l'unico uso \emph{mutable} di variabili non "locali" al thread \`e la fase di assegnazione del name binding tra i nodi dello Stack Graph. Si \`e scelto di non parallelizzare il processo all'interno del prototipo per semplicit\`a, ma \`e comunque fattibile senza troppe modifiche del codice.

%\paragraph{Language Dependency}

Nonostante non sia necessaria una libreria per ogni linguaggio, trascurando le varie librerie che implementano il parser Tree Sitter, \`e comunque un compito gravoso dover analizzare i linguaggi di programmazione da supportare ed implementare le varie grammatice TSG. La grammatica d'esempio \`e per Java e si pu\`o notare nel repository del prototipo quanto sia imponente. Chiaramente rimane il fatto di aver astratto gran parte dell'algoritmo di costruzione del grafo delle dipendenze in modo da essere sufficientemente certi da non dover mettere mano in modo sistematico al codice per implementare il supporto per un nuovo linguaggio.

%\paragraph{Conflitto di Reference Kind}

Una reference viene identificata come da risolvere in base al suo \emph{refkind}, il quale viene applicato dalla grammatica TSG quando assembla ricorsivamente lo Stack Graph. Esistono situazioni ambigue in cui concorrono pi\`u refkind allo stesso nodo, e non disponendo di una funzione nel DSL che permetta di stabilire n\`e se un nodo abbia un attributo n\`e il valore di un certo attributo \`e virtualmente impossibile risolvere questa problematica. Una soluzione potrebbe essere l'uso di attributi distinti per identificare i vari tipi di reference in modo da non avere conflitti nello stesso contesto, ma questo minerebbe alla scalabilit\`a del modello (code smell Shotgun Surgery) e alla leggibilit\`a della grammatica. Si \`e scelto di intervenire riducendo il pi\`u possibile questa eventualit\`a prestando maggiore attenzione della stesura della grammatica.

%\paragraph{Reference Irrisolvibili}

La risoluzione delle reference avviene tramite la ricerca di simboli all'interno del grafo per trovare dei nodi marcati come definizioni. Questo significa che le reference a simboli di librerie di terze parti o in generale di codice non definito nel sorgente analizzato non potranno mai essere identificati. L'unica soluzione ipotizzabile \`e caricare insieme al codice anche le definizioni in forma di Stack Graph di tutte le librerie dell'ecosistema del linguaggio utilizzate, siano esse di terze parti o builtin. Tuttavia questo violerebbe lo stesso scopo di analizzare il codice sorgente in maniera efficiente e con il minimo overhead possibile. Queste reference presentano anche un ulteriore problema prestazionale: ogni atto di risoluzione intentato da Stack Graph fallir\`a sprecando notevole tempo che potrebbe essere utilizzato per cercare reference veramente risolvibili.

\paragraph{Problemi irrisolti}

%\paragraph{Dimensione dello Stack Graph}

La dimensione totale della grammatica TSG che costruisce lo Stack Graph rispecchia in pieno la dimensione finale di tale grafo. Si \`e notato sperimentalmente come anche con sorgenti di dimensione modesta o banale lo Stack Graph tende ad essere molto grande. Ne consegue che da una parte l'algoritmo di risoluzione delle reference sar\`a esponenzialmente pi\`u lento con l'aumentare lineare della dimensione della code base, dall'altra non sempre sar\`a in grado di risolverle con successo. L'euristica di riconoscimento dei cicli tender\`a a bloccare i percorsi che tornano alla radice del grafo perch\'e troppo sotto percorsi erano passati per quel nodo con le stesse condizioni di partenza. Aumentando la variabile \emph{MAX\_SIMILAR\_PATH\_COUNT} non migliora la situazione, il software finisce in stallo o termina la memoria primaria e viene ucciso dal sistema operativo. Questa \`e la ragione per cui alcune relazioni, nonostante sembrino funzionare nei test individuali, non vengono rilevate nel benchmark da parte del prototipo. L'unico modo per superare questo problema sarebbe reimplementare l'euristica che attualmente \`e implementata nel codice sorgente della libreria Stack Graph, ammesso e non concesso che ne esista una definita perfetta o almeno pi\`u efficace.

%\paragraph{Analisi non Differenziali e Aggiornamenti}

Allo stato attuale la Tree Sitter Stack Graph non permette di generare il grafo usando il differenziale di un file, perch\`e invoca la costruzione del Concrete Syntax Tree con Tree Sitter creandone ogni volta uno diverso. Questo deficit elimina uno dei principali vantaggi dell'uso di Tree Sitter e anche in questo caso l'unica via per risolvere il problema \`e mettere mano al codice della libreria. A distanza di 3 mesi dall'inizio dell'implementazione del prototipo \`e stato rilasciato un aggiornamento della libreria di Stack Graph contenente una modifica all'algoritmo di identificazione dei cicli: la nuova versione mette in crisi le risorse del sistema, non termina l'esecuzione nemmeno in input banali come il benchmark Pruijt et Al utilizzato nello scorso capitolo e non garantisce nemmeno di riuscire ad identificare le reference.

\section{Conclusione}

Alla luce delle problematiche esplicitate in questo capitolo come dei risultati dei test del \textbf{Capitolo 6}, \`e infattibile la costruzione di un software che sfrutti efficientemente Stack Graph per estrarre le dipendenze dal codice sorgente. Non solo costruire le grammatiche \`e un compito gravoso ma i presunti vantaggi prestazionali sono inesistenti se non negativi, sia per mancanze dell'implementazione delle librerie, sia perch\`e questa struttura dati non scala bene come promesso nei paper quando si tratta di analisi di progetti non banali. Sebbene Stack Graph si presti mediamente bene alla ricerca di identificatori di un testo di modeste dimensioni (Github usa con relativo successo questa libreria per evidenziare termini all'interno di un solo file), si \`e visto sul campo quanto sia impraticabile impiegarla per analisi pi\`u complesse: la risoluzione delle reference \`e inaffidabile ed estremamente inefficiente.

\section{Sviluppi Futuri}

Se da una parte \`e stato mostrato come non possa essere impiegato per risolvere le reference, pu\`o comunque essere utile come rappresentazione intermedia nell'ottica di costruire il grafo dei componenti. A questo punto si pu\`o costruire un passo di risoluzione delle reference ad hoc che sar\`a indipendente dal linguaggio (astrazione fatta con Stack Graph) e pi\`u efficiente di una semplice visita di grafo. Il principio base di questa risoluzione non \`e molto diversa da quella attuata da Stack Graph: si risolve una catena di identificatori (es: "oggetto.metodo") ricercando per livelli tra gli oggetti definiti nel grafo delle componenti. La procedura per catena \`e semplice, per ogni identificatore si cerca nello scope (limitazione dall'alto) un oggetto con quel nome, quindi in base al suo tipo semantico possono essere trattati diversamente i restanti identificatori della ricerca. Una premessa importante \`e chiarire che quando si parla di \emph{nome} non si intende il \emph{fully qualified name} ma il nome \textbf{proprio} della componente.

Per prima cosa serve una procedura che permetta di stabilire il punto iniziale della ricerca: questo \`e fatto cercando tra i gemelli della reference (pensata come un nodo figlio dello scope) e permettendo all'algoritmo di fluire verso l'alto sostituendo lo scope con il nodo padre dello scope. Chiaramente se in quest'ultimo passo ci si trovasse nella radice dei componenti (package estremo superiore o radice virtuale del progetto) la ricerca sarebbe interrotta con "insuccesso", e si passerebbe alla prossima reference. Per garantire la risalita e la discesa nei termini di nodi del grafo delle componenti \`e necessario che il collegamento tra nodi figli e padri sia bidirezionale.

\begin{lstlisting}[caption="Pseudocodice della procedura di ricerca senza limite superiore", language=Python]
def identify_going_up(scope: Node, identifier: str):
    while True:
        for child in children(scope):
            if name(child) == identifier:
                return child
        scope = parent(scope)
    return None
\end{lstlisting}

In questa versione semplificata della procedura \emph{identify\_going\_up}, si controlla solo tra i nodi figli dello scope al fine di identificarne uno con lo stesso nome dell'identificatore della catena che si sta processando. Anche in questo caso, se non esiste un nodo figlio la risoluzione registrer\`a un insuccesso.

\begin{lstlisting}[caption="Pseudocodice della procedura di ricerca con scope", language=Python]
def identify_simple(scope: Node, identifier: str):
    for child in children(scope):
        if name(child) == identifier:
            return child
    return None
\end{lstlisting}

Il primo passo \`e identificazione del punto d'inizio della ricerca, quindi ad ogni iterazione di ciclo (una per simbolo della catena di reference) si valuter\`a il tipo semantico del nodo di scope. Qualora ci fossero ulteriori identificatori rimanenti da cercare, e lo scope corrispondesse ad un attributo o ad una variabile, esso \`e sostituito con il tipo della variabile. Per fare questo \`e necessario che le reference di tipo \emph{usesType} vengano risolte prima delle altre, altrimenti non si avrebbe accesso a questa informazione.

\begin{lstlisting}[caption="Pseudocodice della procedura di risoluzione delle reference", language=Python]
def solve(scope: Node, reference: list[str]):
    scope = identify_going_up(reference.pop())
    while len(reference) > 0:
        if kind(scope) == (ATTRIBUTE | FIELD):
            scope = type(scope)
        else:
            scope = identify_simple(scope, reference.pop())
    return scope
\end{lstlisting}

Per poter usare queste procedure l'algoritmo di costruzione del grafo delle dipendenze attualmente implementato deve essere modificato per mantenere una lista delle tuple \emph{(source, reference, kind)}, le quali devono essere successivamente ordinate in base al \emph{kind} per poter risolvere prima le dichiarazioni di tipo, e garantire i collegamente bidirezionali fra nodi nella gerarchia (per ora implementata con le dipendenze verso l'alto \emph{definedBy} e \emph{nestedTo}).
