\chapter{Testing Framework}

In questo capitolo si descrive l'approccio di testing, le librerie e il materiale utilizzato per sviluppare il framework di test e alcuni criteri aggiuntivi di valutazione della qualit\`a relativamente alla soluzione adottata correntemente da Arcan.

\section{Approccio di Testing}

Per validare il processo di costruzione del grafo delle dipendenze ci si deve avvalere di uno o pi\`u meccanismi che permettano di verificare che le features dei linguaggi siano supportate, che tutte le relazioni e i nodi vengano identificati correttamente e che il processo riesca a scalare nella complessit\`a in termini di dimensione del codice.

\section{Scaffolding}

Si necessita la creazione di un meccanismo che verifichi il rispetto di politiche espresse in termini di contratti circa il risultato del grafo delle dipendenze. Questo \`e stato implementato direttamente nel software sviluppato: tramite una opportuna opzione da linea di comando si configura il test.

Ogni file contratto conterr\`a sia vincoli su nodi e relazioni, che indicazioni sui file da analizzare:

\begin{lstlisting}
filepaths:
    - file.java
nodes:
    - name: <qualified-name>
      kind: <defkind>
edges:
    - source: <qualified-name>
      sink: <qualified-name>
      kind: <relationship>
\end{lstlisting}

Il software legge usa i file indicati nel campo \emph{filepaths} per costruire il Grafo delle Dipendenze, quindi verifica uno a uno i vincoli specificati: per ogni vincolo di nodo che esista un nodo con quel nome e quel tipo, per ogni relazione che ne esista una tra i nodi specificati tramite i nomi.

Alla fine del processo di test il software tira le somme delineando il risultato, cosa manca e producendo in output un report collettivo, specialmente utile se si processano pi\`u test con lo stesso comando. Di seguito un esempio di report in formato Markdown:

\putimage{diagrams/06/testingReportExample.png}{"Esempio di Report di un contratto"}{fig:testingReportExample}

Esso appende inoltre nel report una tabella vuota per tenere traccia a mano dei problemi principali che \`e possibile inferire dal report e dalla conoscenza di cosa \`e stato implementato.

\putimage{diagrams/06/problemsReport.png}{"Esempio di Report di un contratto"}{fig:testingReportExample}

\section{Benchmark}

Oltre a test individuali per controllare il supporto delle singole caratteristiche, si fa uso anche di un framework per la verifica dell'identificazione delle dipendenze descritto in un paper di Leo Pruijt \cite{DBLP:journals/spe/PruijtKWB17} dove si tratta di valutazione dei sistemi di analisi statica delle dipendenze del codice.

\putimage{diagrams/06/directDependencies.png}{"Dipendenze dirette da rilevare nel codice"}{fig:directDependencies}

\putimage{diagrams/06/indirectDependencies.png}{"Dipendenze indirette da rilevare nel codice"}{fig:indirectDependencies}

\section{Performace}

Oltre alla correttezza dell'analisi e' opportuno tenere traccia delle performance del software e delle grammatiche che vengono utilizzare per costruire il grafo, quindi non solo di valuta la velocit\`a con cui viene generato ma anche la quantit\`a di memoria che viene consumata e la scalabilit\`a del processo.
Per questo alternare test individuali con test pi\`u massivi per mette di stabilire quanto incide la complessit\`a di un progetto rispetto al tempo e alla spazio impiegati.
