\chapter{Testing Framework}

In questo capitolo si descrive l'approccio di testing, le librerie e il materiale utilizzato per sviluppare il framework di test e alcuni criteri aggiuntivi di valutazione della qualit\`a relativamente alla soluzione adottata correntemente da Arcan, quindi i risultati dei test, una valutazione sulle sue capacit\`a e la comparazione con Arcan stesso.

\section{Approccio di Testing}

Per validare il processo di costruzione del grafo delle dipendenze ci si deve avvalere di uno o pi\`u meccanismi che permettano di verificare che le features dei linguaggi siano supportate, che tutte le relazioni e i nodi vengano identificati correttamente e che il processo riesca a scalare nella complessit\`a in termini di dimensione del codice.

\section{Scaffolding}

Si necessita la creazione di un meccanismo che verifichi il rispetto di politiche espresse in termini di contratti circa il risultato del grafo delle dipendenze. Questo \`e stato implementato direttamente nel software sviluppato: tramite una opportuna opzione da linea di comando si configura il test.

Ogni file contratto conterr\`a sia vincoli su nodi e relazioni, che indicazioni sui file da analizzare:

\begin{lstlisting}
filepaths:
    - file.java
nodes:
    - name: <qualified-name>
      kind: <defkind>
edges:
    - source: <qualified-name>
      sink: <qualified-name>
      kind: <relationship>
\end{lstlisting}

Il software legge usa i file indicati nel campo \emph{filepaths} per costruire il Grafo delle Dipendenze, quindi verifica uno a uno i vincoli specificati: per ogni vincolo di nodo che esista un nodo con quel nome e quel tipo, per ogni relazione che ne esista una tra i nodi specificati tramite i nomi.
Alla fine del processo di test il software tira le somme delineando il risultato, cosa manca e producendo in output un report collettivo, specialmente utile se si processano pi\`u test con lo stesso comando. Esso appende inoltre nel report una tabella Markdown vuota per tenere traccia a mano dei problemi principali che \`e possibile inferire dal report e dalla conoscenza di cosa \`e stato implementato.

L'implementazione del prototipo ha previsto lo sviluppo della grammatica TSG per Java e dei relativi test. Di seguito un esempio di report in formato Markdown di uno dei test:

\putimage{diagrams/06/testingReportExample.png}{"Test della stuite realizzato per Java sulle relazioni di inclusione misto con relazioni di uso per aumentare la complessit\`a e garantire l'usabilit\`a"}{fig:testingReportExample}

Grazie a questo strumento di testing si pu\`o adottare un approccio sistematico nell'implementazione delle grammatiche da una parte, e dell'algoritmo che costruisce il grafo dall'altra. Per tenere traccia delle operazioni e organizzare le implementazioni si pu\`o fare uso di uno strumento di Version Control come Git, specialmente se si lavora in team in parallelo. Nel caso corrente \`e stato utilizzato per gestire la versione in produzione affinch\`e non avesse implementazioni sperimentali o parziali.

\begin{lstlisting}[caption="Processo di implementazione", language=Python]
for language in SUPPORTED_LANGUAGES:
    # creazione ipotetica del fork del branch di sviluppo
    branch = forkBranch("development")
    # elencazione per enumerazione di tutte le features
    # del linguaggio di programmazione che siano utili
    # ai fini delle relazioni da identificare
    # o che richiedano approcci ad hoc
    features = findLanguageFeatures()
    for feature in features:
        # costruzione di casi di test come coppia (sorgente, politica);
        # un numero di test adeguato rispetto
        # alla variabilit\`a del costrutto e
        # alle possibili implicazioni rispetto
        # ad altre relazioni gi\`a sviluppate
        buildTestFor(language, feature)
        # implementazione nella grammatica TSG e nel codice rust per introdurre
        # la rilevazione di nodi e archi del grafo delle dipendenze
        implementFeature(language, feature)
        # esecuzione di tutti i test finora implementati per assicurarsi che
        # la nuova feature non vada a interferire con le altre gi\`a presenti
        integrationTest()
        # salvare iterativamente i progressi permette di garantire
        # la presenza di un punto di ritorno sicuro
        # nel caso si commetta qualche errore durante lo sviluppo
        commitBranch(branch)
    joinBranch("development", branch)
\end{lstlisting}

Di seguito la tabella dei test creata per il linguaggio Java con l'indicazione delle relazioni e delle feature testate.

\begin{center}
    \captionof{table}{Test per Java sulle dipendenze e i loro relativi esiti}
    \begin{tabular}{|l l c|}
        \hline
        Test & Relazioni & Risultato \\
        \hline
        implementation\_bridge & calls & \cmark \\
        class\_methods\_with\_parameters & definedBy & \cmark \\
        class\_implementation & isImplementationOf & \cmark \\
        class\_with\_attributes & definedBy & \cmark \\
        class\_methods\_with\_attributes & definedBy & \cmark \\
        class\_method\_call & calls & \cmark \\
        type\_inference & calls, accessField & \cmark \\
        enums & definedBy, accessField & \cmark \\
        class\_inheritance & isChildOf, nestedTo & \cmark \\
        object\_creation & usesType & \cmark \\
        class\_type\_usage\_nested\_packages & definedBy, usesType & \cmark \\
        nested\_classes & nestedTo & \cmark \\
        casts\_type & castsType & \cmark \\
        class\_type\_usage & definedBy, usesType & \cmark \\
        throws\_type & throwsType & \cmark \\
        imports & definedBy, includes, usesType & \cmark \\
        interface\_inheritance & isChildOf & \cmark \\
        type\_inference\_with\_packages & calls, accessField & \cmark \\
        extension\_bridge & calls & \cmark \\
        class\_constructors & definedBy & \cmark \\
        class\_field\_access & accessField & \cmark \\
        class\_packages & definedBy, includes & \cmark \\
        annotation & usesType & \cmark \\
        array\_creation & usesType & \cmark \\
        class\_inheritance\_with\_packages & isChildOf & \cmark \\
        class\_methods & definedBy & \cmark \\
        \hline
    \end{tabular}
\end{center}


\section{Benchmark}

Oltre a test individuali per controllare il supporto delle singole caratteristiche, si fa uso anche di un framework per la verifica dell'identificazione delle dipendenze descritto in un paper di Pruijt et al \cite{DBLP:journals/spe/PruijtKWB17} dove si tratta di valutazione dei sistemi di analisi statica delle dipendenze del codice.

\begin{center}
    \captionof{table}{Direct dependency types in the Benchmark test}
    \begin{tabular}{|l l|}
        \hline
        Dependency type & Example code \\
        \hline
        \textbf{Import} &  \\
        Class import & import ModuleB.ModuleB2.Class3; \\
        \hline
        \textbf{Declaration} &  \\
        Instance variable & private Class3 class3; \\
        Class variable & private static Class3 class3; \\
        Local variable & public void method() {Class3 class3; } \\
        Parameter & public void method(Class3 class3) {} \\
        Return type & public Class3 method() {} \\
        Exception & public void method() throws Class4{throw new Class4 ("..."); } \\
        Type cast & Object o = (Class3) new Object(); \\
        \hline
        \textbf{Call} &  \\
        Instance method & variable = class3.method(); \\
        Instance method-inherited & variable = class3.methodSuper(); \\
        Class method & variable = class3.classMethod(); \\
        Constructor & new Class3(); \\
        Inner class & method variable = class3.InnerClass.method(); \\
        Interface method & interface1.interfaceMethod(); \\
        Library class method & libraryClass1.libraryMethod(); \\
        \hline
        \textbf{Access} &  \\
        Instance variable & variable = class3.variable; \\
        Instance variable-inherited & variable = class3.variableSuper; \\
        Class variable & variable = Class3.classVariable; \\
        Constant variable & variable = class3.constantVariable; \\
        Enumeration & System.out.println(Enumeration.VAL1); \\
        Object reference & method(class3); \\
        \hline
        \textbf{Inheritance} &  \\
        Extends class & public class Class1 extends Class3 { } \\
        Extends abstract & class Idem but in this case Class3 should be abstract. \\
        Implements interface & public class Class1 implements Interface1 { } \\
        \hline
        \textbf{Annotation} &  \\
        Class annotation & @Class3 \\
        \hline
    \end{tabular}
\end{center}

\begin{center}
    \captionof{table}{Indirect dependency types in the Benchmark test}
    \begin{tabular}{|l l|}
        \hline
        Dependency type & Example code \\
        \hline
        \textbf{Call} &  \\
        Instance method & variable = class2.class3.method(); \\
        Instance method-inherited & variable = class2.methodSuper(); \\
        Class method & variable = class2.class3.classMethod(); \\
        \hline
        \textbf{Access} &  \\
        Instance variable & variable = class2.class3.variable; \\
        Instance variable-inherited & variable = class2.variableSuper(); \\
        Class variable & variable = class2.class3.classVariable; \\
        Object reference-Reference var & variable = class2.method(class2.class3.class4); \\
        Object reference-Return value & Object o = (Object) class2.getClass4(); \\
        \hline
        \textbf{Inheritance} &  \\
        Extends-implements variations & public class Class1 extends Class2 { } \\
         & public class Class2 extends SuperClass { } \\
        \hline
    \end{tabular}
\end{center}

Da queste tabelle il team di Arcan ha costruito una code base di esempio e quindi estratto una checklist di dipendenze legate alla code base che pu\`o essere utilizzata per accertare la capacit\`a di un tool di rilevare correttamente tutte le principali dipendenze del codice. Nella fase di validazione del software descritto in questi capitolo si fa uso della code base Java e della checklist, quest'ultima opportunamente tradotta in relazioni rilevabili dal modello in esame. Di seguito i risultati del benchmark in formato di tabella:

\begin{center}
    \captionof{table}{Risultati del Benchmark del Prototipo e di Arcan}
    \begin{tabular}{|l l l | c | c|}
        \hline
        Sorgente & Destinazione & Relazione & Prototipo & Arcan \\
        \hline
        \textbf{Access} &&&& \\
        AccessClassVariable & CheckInDAO & accessField & \cmark & \cmark \\
        AccessClassVariableConstant & UserDAO & accessField & \cmark & \cmark \\
        AccessClassVariableInterface & ISierraDAO & accessField & \cmark & \cmark \\
        AccessEnumeration & TipDAO & accessField & \cmark & \cmark \\
        AccessInstanceVariableRead & ProfileDAO & accessField & \xmark & \cmark \\
        AccessInstanceVariableWrite & ProfileDAO & accessField & \xmark & \cmark \\
        AccessInstanceVariableConstant & UserDAO & accessField & \xmark & \cmark \\
        AccessInstanceVariableSuperClass & CallInstanceSuperClassDAO & accessField & \xmark & \xmark \\
        AccessInstanceVariableSuperSuperClass & CallInstanceSuperClassDAO & accessField & \xmark & \xmark \\
        AccessObjectReferenceAsParameter & Base & accessField & \cmark & \xmark \\
        AccessObjectReferenceWithinIfStatement & Base & accessField & \cmark & \xmark \\
        \textbf{Annotations} &&&& \\
        AnnotationDependency & SettingsAnnotation & usesType & \cmark & \cmark \\
        \textbf{Call} &&&& \\
        CallClassMethod & BadgesDAO & calls & \cmark & \cmark \\
        CallConstructor & AccountDAO & usesType & \cmark & \cmark \\
        CallInstance & ProfileDAO & calls & \xmark & \xmark \\
        CallInstanceInnerClass & CallInstanceOuterClassDAO & calls & \xmark & \xmark \\
        CallInstanceInterface & CallInstanceInterfaceDAO & calls & \xmark & \xmark \\
        CallInstanceSuperClass & CallInstanceSuperClassDAO & calls & \xmark & \cmark \\
        CallInstanceSuperSuperClass & CallInstanceSuperClassDAO & calls & \xmark & \cmark \\
        \textbf{Declaration} &&&& \\
        DeclarationExceptionThrows & StaticsException & throwsType & \cmark & \cmark \\
        DeclarationParameter & ProfileDAO & usesType & \cmark & \cmark \\
        DeclarationReturnType & VenueDAO & usesType & \cmark & \cmark \\
        DeclarationTypeCast & ProfileDAO & castsType & \cmark & \cmark \\
        DeclarationTypeCastOfArgument & ProfileDAO & castsType & \cmark & \cmark \\
        DeclarationVariableInstance & ProfileDAO & usesType & \cmark & \cmark \\
        DeclarationVariableLocal & ProfileDAO & usesType & \cmark & \cmark \\
        DeclarationVariableLocal\_Initialized & ProfileDAO & usesType & \cmark & \cmark \\
        DeclarationVariableStatic & ProfileDAO & usesType & \cmark & \cmark \\
        \textbf{Import} &&&& \\
        domain.direct.violating & AccountDAO & includes & \cmark & \xmark \\
        \textbf{Inheritance} &&&& \\
        InheritanceExtends & HistoryDAO & isChildOf & \cmark & \cmark \\
        InheritanceExtendsAbstractClass & FriendsDAO & isChildOf & \cmark & \cmark \\
        InheritanceImplementsInterface & IMapDAO & isImplementationOf & \cmark & \cmark \\
        \hline
    \end{tabular}
\end{center}

Per semplicit\`a di lettura sono indicate la classi o i package che fanno riferimento alla relazione invece delle componenti precise, tuttavia i dati completi sono disponibili nel repository del prototipo nel file \emph{report\_java.md} \cite{SkullianRepository}. Il prototipo rileva correttamente solo 22 delle 32 dipendenze totali presenti nel benchmark, a fronte delle 24 rilevate da Arcan. Tuttavia si deve notare come nel caso del prototipo le dipendenze non appartengono ad una dipendenza ombrello "dependsOn" come nel caso di Arcan, ma ogni label \`e descrittiva della dipendenza che sussiste tra i due componenti. Un'analisi del problema \`e presente nel capitolo conclusivo.

\section{Performace}

Oltre alla correttezza dell'analisi \`e opportuno tenere traccia delle performance del software e delle grammatiche che vengono utilizzare per costruire il grafo, quindi non solo di valuta la velocit\`a con cui viene generato ma anche la quantit\`a di memoria che viene consumata e la scalabilit\`a del processo.
Per questo alternare test individuali con test pi\`u massivi per mette di stabilire quanto incide la complessit\`a di un progetto rispetto al tempo e alla spazio impiegati. La verifica delle prestazioni \`e effettuata tramite l'uso del comando unix \emph{time} come prologo del comando di lancio dei due software.
Entrambi i test sono stati eseguiti nelle medesime condizioni sullo stesso hardware e sistema operativo: un portatile \emph{HP Pavilion Laptop 15-eg0xxx} con CPU \emph{11th Gen Intel i5-1135G7 (4) @ 2.420GHz}, GPU \emph{Intel(R) Iris(R) Xe Graphics}, RAM \emph{4 GB}, e sistema operativo \emph{Linux} distribuzione \emph{Arch Linux} con gli ultimi aggiornamenti possibili installati.

Eseguendo entrambi i software su 5 progetti java si pu\`o notare come il prototipo sia significativamente pi\'u lento di Arcan nel caso medio. Tanto che nel caso di fastjson e guava viene molto rallentato dalla moltitudine di reference "a vuoto", (di cui si parler\`a nel prossimo capitolo) e finisce per impiegare un tempo irragionevole. \`E anche bene considerare che il primo \`e s\`i scritto in un linguaggio di programmazione pi\`u efficiente del secondo, ma \`e anche privo di parallelizzazione: eventualmente sarebbe possibile aumentarne l'efficienza parallelizzando con un fatto pari a quello di Arcan, ma non influirebbe di molto se non nei casi pi\`u piccoli.
Infine non \`e stato possibile valutare in modo oggettivo il tempo di eseguzione di Arcan su guava perch\'e, sebbene segnali il termine dell'operazione circa dopo 2 minuti di esecuzione, il processo non termina regolarmente ma i thread java rimangono attivi in background.

\begin{center}
    \captionof{table}{Tempi di analisi (in secondi) di vari progetti da parte del Prototipo e di Arcan}
    \begin{tabular}{|l|l|l|l|l|}
        \hline
        progetto & software & tempo minimo & tempo massimo & tempo medio \\ \hline
        \hline
        \rowcolor[HTML]{FE0000}
        guava    & Prototipo & N/A    & N/A    & N/A    \\ \hline
        \rowcolor[HTML]{34FF34} 
        guava    & Arcan     & N/A    & N/A    & \~120   \\ \hline
        \rowcolor[HTML]{F27805} 
        junit4   & Prototipo & 65.82  & 69.07  & 65.88  \\ \hline
        \rowcolor[HTML]{34FF34} 
        junit4   & Arcan     & 13.850 & 17.079 & 14.611 \\ \hline
        \rowcolor[HTML]{F27805} 
        junit5   & Prototipo & 132.87 & 134.75 & 134.44 \\ \hline
        \rowcolor[HTML]{34FF34} 
        junit5   & Arcan     & 42.542 & 47.613 & 44.186 \\ \hline
        \rowcolor[HTML]{F27805} 
        antlr4   & Prototipo & 171.65 & 175.49 & 172.64 \\ \hline
        \rowcolor[HTML]{34FF34} 
        antlr4   & Arcan     & 19.222 & 20.140 & 19.691 \\ \hline
        \rowcolor[HTML]{FE0000} 
        fastjson & Prototipo & N/A    & N/A    & N/A    \\ \hline
        \rowcolor[HTML]{34FF34} 
        fastjson & Arcan     & 66.932 & 71.094 & 69.071 \\ \hline
    \end{tabular}
\end{center}

\begin{sloppypar}
A proposito di questi progetti \`e interessante rilevare la similarit\`a dei grafi costruiti dai due software. Per fare questo ci si \`e serviti dell'indice di Jaccard, definito come $\frac {A \cap B} {A \cup B}$, ovvero la percentuale di elementi comuni tra due insiemi rispetto al totale degli elementi dell'unione. \`E stato necessario ridurre il grafo del prototipo in una forma comparabile con quello di Arcan: gli unici nodi considerati sono di tipo \emph{annotation}, \emph{class}, \emph{enum}, \emph{interface}, \emph{package}, mentre per le relazioni non verranno considerate due dipendenze di Arcan: \emph{unitIsAfferentOf} e \emph{containerIsAfferentOf}. Le relazioni tra nodi che non sono considerati vengono trasferite ai nodi avi considerati. Tutte le dipendenze vengono considerate uguali indipendentemente dalla label associata, visto che di base Arcan non effettua quasi distinzione.
\end{sloppypar}
\`E osservabile come la differenza tra l'accuratezza dei due software sia abissale (eccetto \emph{antlr4} dove comunque Arcan mantiene una schiacciante superiorit\`a), la qualit\`a dell'analisi del prototipo non giustifica minimamente la differenza prestazionale.

\begin{center}
    \captionof{table}{Indici di Jaccard su nodi, archi e grafi prodotti dal Prototipo e da Arcan su vari progetti}
    \begin{tabular}{|l|l|l|l|}
        \hline
        progetto & Jaccard(nodi) & Jaccard(archi) & Jaccard(grafi) \\ \hline
        \hline
        \rowcolor[HTML]{FE0000} 
        junit4 & 37.26\% & 23.19\% & 25.42\% \\ \hline
        \rowcolor[HTML]{FE0000} 
        junit5 & 47.41\% & 15.13\% & 19.73\% \\ \hline
        \rowcolor[HTML]{F27805} 
        antlr4 & 82.18\% & 44.25\% & 48.43\% \\ \hline
    \end{tabular}
\end{center}

