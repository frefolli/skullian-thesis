\chapter{Dependency Graph}

In questo capitolo si descrive l'algoritmo di costruizione del Dependency Graph nelle sue varie fasi.

\section{Il Grafo}

Il Grafo delle Dipendenze deve essere modellato in modo da essere sovrapponibile a quello prodotto da Arcan.
In questo senso si attua una semplificazione del modello appiattendo la distinzione tra Unit, Container e Module: i nodi saranno distinti dall'attributo defkind che \`e stato trattato nei capitoli precedenti.
Per quanto riguarda invece le relazioni, esse saranno all'incirca identiche, seppur con qualche distinzione nel merito delle entit\`a coinvolte.

Sia $G = (V, E)$ un grafo delle dipendenze, il generico nodo $v \in V$ sar\`a etichettato con il suo \emph{fully qualified name} e avr\`a associato il \emph{defkind}.
Quindi il generico arco $e \in E$ sar\`a etichettato con il nome della relazione che esso rappresenta.

\section{Costruzione}

La costruzione avver\`a in tre fasi.
Prima di tutto si usa Tree Sitter Stack Graph per costruire lo Stack Graph del codice usando una grammatica costruita con il modello del Capitolo 4.
Quindi si proceder\`a a risolvere le reference dello Stack Graph: questa operazione \`e molto importante in quanto permetter\`a a una esplorazione successiva di sostituire immediatamente i nodi di reference delle relazioni con i nodi concreti di definizione e quindi costruire relazioni con i nodi del grafo delle dipendenze giusti.
L'ultimo passo sar\`a una esplorazione ricorsiva che permetter\`a di evidenziare tutti i nodi di definizione e costruire le relazioni di parentela e quelle dettate dalle reference.
Al termine dell'operazione sar\`a necessario poter esportare il grafo per permettere a ulteriore software di fare elaborazioni successive, come per esempio Arcan.

\subsection{Analisi del Codice Sorgente}

Il software legge dalla linea di comando una lista di \emph{target}, ovvero la lista dei progetti da analizzare.
Ogni target pu\`o essere un file singolo o una cartella, nel secondo caso si considereranno tutti i file di essa ricorsivamente. Se il target \`e una cartella ci si sposta all'interno di essa, per ridurre la complessit\`a dei \emph{fully qualified name} per le grammatiche che necessitano dell'unit\`a \emph{File}.
Si itera quindi su tutti i target, ed in particolare su tutti i file contenuti ricorsivamente nei target che sono cartelle, costruendo incrementalmente lo Stack Graph.

Per semplicit\`a si pu\`o considerare l'analisi di progetti costituiti di un solo linguaggio di programmazione, ma il processo si adatta anche alla possibilit\`a di avere progetti misti. Prima di costruire lo Stack Graph \`e necessario caricare la grammatica TSG che Tree Sitter Graph user\`a per costruirlo: in Rust questa \`e una operazione onerosa, quindi si rende necessario l'uso di una \emph{cache} che non solo permetta di caricare dinamicamente e riutilizzare le grammatiche ma si presti bene a scalare il numero di linguaggi utilizzati nel progetto.

\begin{lstlisting}[language=Python, caption="Pseudocodice della procedura build_stack_graph"]
def build_stack_graph(config):
    # inizializzazione delle strutture dati
    stack_graph = StackGraph.new()
    tsg_cache = TSGCache.new()
    # iterazione sui target
    for target in config.targets:
        if is_directory(target):
            change_directory(target)
        # iterazione sui file
        for file in all_files_from(target):
            language = get_programming_language(file)
            tsg_grammar = get_tsg_grammar(tsg_cache, language)
            stack_graph.extend(file, tsg_grammar)
    return stack_graph
\end{lstlisting}

\subsection{Risoluzione delle Reference}

Quindi si passa alla fase di risoluzione delle reference, ma verranno processate solo i nodi di reference con un \emph{refkind} valido, visto che nel modello inevitabilmente si introducono dei nodi che hanno solo lo scopo di aiutare la costruzione delle unit\`a e poi la risoluzione delle reference.
Per fare questo si pu\`o iterare su tutti i nodi dello Stack Graph e controllare questi requisiti.
Ogni nodo identificato in questo modo sar\`a passato alla procedura di risoluzione manuale introdotta nel Capitolo 2 nella sezione "In Rust".

Per mantenere la lista delle risoluzioni avvenute con successo si utilizza una mappa che associa il nodo di reference al nodo di definizione corrispondente: sar\`a usata pi\`u avanti per sostituire velocemente i nodi esplorati.
Visto che lo scopo \`e instaurare relazioni tra due nodi del Grafo delle Dipendenze e che i nodi di definizione con \emph{defkind} saranno nodi concreti di tale grafo, si aggiunge come requisito che il nodo trovato con la risoluzione abbia un defkind valido.

\begin{lstlisting}[language=Python, caption="Pseudocodice della procedura resolve_references"]
def resolve_references(stack_graph):
    # inizializzazione
    map: Map<Node, Node>.new()
    # iterazione sui nodi
    for node in stack_graph.get_nodes():
        # controllo dei requisiti
        if is_reference(node):
            if has_refkind(node):
                definition = resolve_reference(stack_graph, node)
                # controllo di avvenuta risoluzione
                # e dei requisiti del nodo trovato
                if is_some(definition):
                    if has_defkind(definition):
                        map.set(node, definition)
    return map
\end{lstlisting}

\subsection{Esplorazione Ricorsiva}

L'esplorazione sar\`a una visita Depth First \cite{cormen2009introduction} a partire dalla radice dello Stack Graph, e per ogni nodo non ancora visitato si distinguer\`a in base al suo tipo.

Per stabilire relazioni di parentela nel codice \`e sufficiente badare all'ordine di esplorazione dei nodi: i nodi esplorati dopo un certo nodo possono essere considerati in questo modello "figli", quindi \`e necessario tenere traccia dell'ultimo nodo visitato prima del corrente. Essendo lo Stack Graph aciclico, salvo su nodi che necessariamente sono gi\`a stati esplorati, si \`e sicuri che le relazioni di parentela assegnate in tal modo siano corrette.

Oltre alla gi\'a citata relazione "definedBy" si user\'a anche la relazione "nestedTo": la prima quando i due nodi hanno defkind differente, l'altra nel caso opposto.

\begin{lstlisting}[language=Python, caption="Pseudocodice della procedura walk_step"]
def walk_step(reference_map, dependency_graph,
              parent_node, current_node):
    if not_visited(current_node):
        visit(current_node)
        if is_definition(current_node) and has_defkind(current_node):
            walk_step_definition(reference_map, dependency_graph,
                                 parent_node, current_node)
        elif is_reference(current_node) and has_refkind(current_node):
            walk_step_reference(reference_map, dependency_graph,
                                 parent_node, current_node)
        else:
            walk_step_other(reference_map, dependency_graph,
                            parent_node, current_node)
def build_dependency_graph(stack_graph, reference_map):
    dependency_graph = DependencyGraph.new()
    walk_step(reference_map, dependency_graph,
                     None, stack_graph.get_root())
    return dependency_graph
\end{lstlisting}

\paragraph{definition}

Da un nodo di definizione con un \emph{defkind} valido si estrae il simbolo che rappresenta e si usa esso per calcolare il \emph{fully qualified name} concatenando il nome del nodo "padre" con il nome del nodo "figlio". Visto che i File non sono costrutti specifici di un linguaggio di programmazione si user\`a come delimitatore la stringa "::" invece del carattere ".".

Quindi si crea il nodo del grafo in costruzione usano il nome composto e il defkind. Esistono linguaggi dove si possono trovare definizioni duplicate legittime, come per esempio i Package in Java, quindi si controlla prima che non ne esista gi\'a uno all'interno del grafo delle dipendenze.

Come ultima cosa si crea la relazione di parentela se il nodo "padre" \`e definito non nullo e si applica il passo ricorsivo.
Esso consiste nell'esplorare tutti i nodi collegati con archi a esso usando come nodo padre il nodo del grafo delle dipendenze appena creato.

\begin{lstlisting}[language=Python, caption="Pseudocodice della procedura walk_step_definition"]
def walk_step_definition(reference_map, dependency_graph,
                         parent_node, current_node):
    defkind = get_defkind(current_node)
    name = concat(parent_node, current_node)
    # prevenzione della duplicazione dei nodi
    if dependency_graph.contains(name):
        depgraph_node = dependency_graph.get(name)
    else:
        depgraph_node = dependency_graph.newNode(name, defkind)
    # creazione relazioni definedBy e nestedTo
    if some(parent_node):
        if get_defkind(parent_node) == defkind:
            dependency_graph.newEdge(depgraph_node,
                                     parent_node,
                                     "nestedTo")
        else:
            dependency_graph.newEdge(depgraph_node,
                                     parent_node,
                                     "definedBy")
    # passo ricorsivo
    for next_node in current_node.get_edges():
        walk_step(reference_map, dependency_graph,
                  depgraph_node, next_node)
\end{lstlisting}

\paragraph{reference}

Se un nodo di reference ha nella mappa delle reference risolte in precedenza un'associazione a un nodo di definizione, si associa una relazione di quel tipo (tradotta sia per motivi di leggibilit\`a, che per uniformarla alla notazione di Arcan) tra il nodo trovato e il nodo "padre".

Quindi si applica una variante del passo ricorsivo dei nodi di definizione: l'unica differenza sta nella conservazione del nodo padre.

\begin{lstlisting}[language=Python, caption="Pseudocodice della procedura walk_step_reference"]
def translate(refkind):
    if refkind == "Extends":
        return "isChildOf"
    elif refkind == "Implements":
        return "isImplementationOf"
    else:
        return refkind
def walk_step_reference(reference_map, dependency_graph,
                        parent_node, current_node):
    refkind = get_refkind(current_node)
    if reference_map.has(current_node):
        definition = reference_map.get(current_node)
        dependency_graph.newEdge(parent_node,
                                 current_node,
                                 translate(refkind))
    # passo ricorsivo
    for next_node in current_node.get_edges():
        walk_step(reference_map, dependency_graph,
                  parent_node, next_node)
\end{lstlisting}

\paragraph{altro}

In tutti gli altri casi il nodo esplorato deve essere considerato come un nodo di transito: si applica solo il passo ricorsivo descritto per i nodi di reference.

\begin{lstlisting}[language=Python, caption="Pseudocodice della procedura walk_step_other"]
def walk_step_other(reference_map, dependency_graph,
                         parent_node, current_node):
    # passo ricorsivo
    for next_node in current_node.get_edges():
        walk_step(reference_map, dependency_graph,
                  parent_node, next_node)
\end{lstlisting}

\section{Esportazione}

In questa sezione vengono trattati i tre principali formati utilizzati durante lo sviluppo del software e delle grammatiche TSG per visualizzare i progressi, per ricevere un feedback sullo stato del lavoro, e che possono essere utilizzati anche come mezzo di condivisione dei risultati dell'analisi delle dipendenze.

\subsection{PlantUML}

PlantUML \cite{PlantUML} \`e un software FOSS utilizzato per generare diagrammi e grafici a partire da una descrizione degli stessi in plain text.

Prima si produce la lista dei nodi accompagnandoli con il rispettivo \emph{defkind}, quindi si inseriscono le relazioni con una sintazzi molto intuitiva:

\begin{lstlisting}
;; esempio di nodo
(fully_qualified_name) <<defkind>>
;; esempio di relazione
(classeA) ----> (packageB) : definedBy
\end{lstlisting}

\subsection{Cytoscape}

Cytoscape \cite{CytoscapeJS} \`e un framework per la visualizzazione di grafi e reticoli che utilizza JSON come formato di markup per la rappresentazione di nodi ed archi.

Il file prodotto consiste essenzialmente di un array di oggetti contenenti nodi o archi: non c'\`e un ordine prestabilito. Un esempio di rappresentazione \`e il seguente:

\begin{lstlisting}[language=JSON, caption=esempio di nodo]
{
    "data": {
        "id": "0",
        "label": "name",
        "kind": "defkind"
    }
}
\end{lstlisting}

\begin{lstlisting}[language=JSON, caption=esempio di arco]
{
    "data": {
        "id": "1 -> 0",
        "label": "extends",
        "kind": "extends",
        "source": "1",
        "target": "0"
    }
}
\end{lstlisting}

\subsection{GraphML}

Con GraphML \cite{GraphML} si intende un formato standard per la rappresentazione dei grafi che usa XML come base e si prefigge di essere semplice e generale.

La struttura del documento \`e simile a questa:

\begin{lstlisting}[language=XML, caption=file GraphML]
<?xml version=\"1.0\" encoding=\"UTF-8\"?>
<graphml xmlns=\"http://graphml.graphdrawing.org/xmlns\"
         xmlns:xsi=\"http://www.w3.org/2001/XMLSchema-instance\"
         xsi:schemaLocation=\"http://graphml.graphdrawing.org/xmlns
            http://graphml.graphdrawing.org/xmlns/1.0/graphml.xsd\">
    <key id=\"name\" for=\"node\"
         attr.name=\"name\" attr.type=\"string\">
        <default>yellow</default>
    </key>
    <key id=\"kind\" for=\"node\"
         attr.name=\"kind\" attr.type=\"string\">
        <default>yellow</default>
    </key>
    <key id=\"relationship\" for=\"edge\"
         attr.name=\"relationship\" attr.type=\"string\"/>
    <graph id=\"dep-graph\" edgedefault=\"directed\">
        <!-- dati del grafo -->
    </graph>
</graphml>
\end{lstlisting}

Ogni elemento del tag \emph{graph} sar\`a un nodo o un arco:

\begin{lstlisting}[language=XML, caption=esempio di nodo]
<node id=\"0\">
    <data key=\"name\">name</data>
    <data key=\"kind\">defkind</data>
</node>
\end{lstlisting}

\begin{lstlisting}[language=XML, caption=esempio di arco]
<edge source=\"0\" target=\"1\">
    <data key=\"relationship\">relationship</data>
</edge>
\end{lstlisting}

\section{Esempio}

Sia dato il seguente codice d'esempio Java:

\begin{lstlisting}[language=Java]
package unimib.ingsof;
class Animal {
  String name;
  public void eat() {
    System.out.println("I can eat");
  }
}
class Dog extends Animal {
  public void display() {
    System.out.println("My name is " + name);
  }
}
class Main {
  public static void main(String[] args) {
    Dog labrador = new Dog();
    labrador.name = "Rohu";
    labrador.display();
    labrador.eat();
  }
}
\end{lstlisting}

Il procedimento di costruzione presentato in questo capitolo deve essere in grado di costruire il seguente Grafo delle Dipendenze:

\putimage{diagrams/05/depGraphExample.png}{"Esempio di Grafo delle Dipendenze"}{fig:depGraphExample}

In questa immagine le componenti sono renderizzate con Cytoscape.js, una libreria Javascript adatta alla visualizzazione di grafi, in modo da distinguere i nodi in base al \emph{defkind} con forma e colore: quindi per esempio i quadrati rossi sono di tipo Method e gli esagono blu sono di tipo Class.
Si pu\`o notare come oltre alle reference gi\`a citate si tiene traccia anche della relazione di parentela \emph{definedBy}, che mette in relazione il codice sorgente contenuto con il contenitore.
