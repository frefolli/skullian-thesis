\chapter{Dependency Graph}

In questo capitolo si descrive l'algoritmo di costruizione del Dependency Graph nelle sue varie fasi.

\section{Il Grafo}

Il Grafo delle Dipendenze deve essere modellato in modo da essere sovrapponibile a quello prodotto da Arcan.
In questo senso si attua una semplificazione del modello appiattendo la distinzione tra Unit, Container e Module: i nodi saranno distinti dall'attributo defkind che e' stato trattato nei capitoli precedenti.
Per quanto riguarda invece le relazioni, esse saranno all'incirca identiche, seppur con qualche distinzione nel merito delle entita' coinvolte.

Sia $G = (V, E)$ un grafo delle dipendenze, il generico nodo $v \in V$ sara' etichettato con il suo \emph{fully qualified name} e avra' associato il \emph{defkind}.
Quindi il generico arco $e \in E$ sara' etichettato con il nome della relazione che esso rappresenta.

\section{Esempio}

Sia dato il seguente codice d'esempio Java:

\begin{lstlisting}[language=Java]
package unimib.ingsof;

class Animal {
  String name;
  public void eat() {
    System.out.println("I can eat");
  }
}

class Dog extends Animal {
  public void display() {
    System.out.println("My name is " + name);
  }
}

class Main {
  public static void main(String[] args) {
    Dog labrador = new Dog();
    labrador.name = "Rohu";
    labrador.display();
    labrador.eat();
  }
}
\end{lstlisting}

Il procedimento di costruzione presentato in questo capitolo deve essere in grado di costruire il seguente Grafo delle Dipendenze:

\putimage{diagrams/05/depGraphExample.png}{"Esempio di Grafo delle Dipendenze"}{fig:depGraphExample}

\section{Costruzione}

\paragraph{Name Resolution}
%TODO

\paragraph{Esplorazione Ricorsiva}
%TODO

\section{Esportazione}
%TODO
