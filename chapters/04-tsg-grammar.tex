\chapter{Grammatica TSG}

In questo capitolo viene descritto il modello da implementare per costruire una grammatica TSG che sia funzionale sia a costruire il grafo delle dipendenze che a risolvere le reference con Stack Graph.

\section{Il Modello}

\subsection{Language Dependency}
%TODO: quote

Come visto nel capitolo precedente, per costruire lo Stack Graph per un certo file sorgente si utilizza un file con alcune regole che descrivono a Tree Sitter Stack Graph come costruirlo.

\par
Tuttavia bisogna ricordare che l'albero sintattico di due linguaggi diversi allo stato attuale in Tree Sitter ha non solo nomi diversi per i nodi, ma anche struttura interentemente diversa, sia perche' le rispettive sintassi possono essere molto diverse e quindi necessitano di essere trattate diversamente, sia perche', quando anche la loro sintassi fosse simile o uguale, i relativi sviluppatori le trattano in modo diverso o con passaggi intermedi (C preprocessor) o con semantica diversa.

\par
Se quindi da una parte le query delle regole saranno necessariamente molto diverse, bisogna considerare che i linguaggi di programmazione sono eterogenei anche nei costrutti.
Rust, per esempio, non possiede ne' classi ne' interfacce. Un concetto simile a quest'ultime sono i \textbf{Traits}, ma vengono applicati secondo una filosofia \emph{Composition over Inheritance}.
La mancanza o la rielaborazioni di costrutti e componenti nei linguaggi e' anche quindi strettamente legata ai loro paradigmi, ma possono sussistere differenze molto piu' sottili che pero' necessitano di dovute differenze nelle grammatiche TSG: si pensi per esempio ai nomi dei package, problema gia' citato nel capitolo precedente.

\par
Non potendo scrivere un'unica grammatica TSG per ogni linguaggi l'approccio a questo problema deve essere sistematico: si elabora un modello per componenti, il piu' generale possibile, e si implementa tale modello nei vari linguaggi che si ritiene utile supportare.
Questo approccio permette di limitare la dipendenza da software di terze parti (si pensi ai Language Server) e allo stesso tempo di evitare soluzioni standalone specifiche per ogni linguaggio, creando un solido strato di astrazione che permetta anche agli algoritmi piu' sofisticati di operare ed analizzare gli Stack Graph dei sorgenti uniformando componenti, relazioni e struttura.

\subsection{Algoritmo di Name Resolution di Stack Graph}

E' importante comprendere come Stack Graphs effettivamente risolve le reference perche' non solo abilita il programmatore ad ottimizzazioni consapevoli, ma permette anche di implementare risoluzioni piu' complesse.

\par
Sia dato un grafo $G = (V, E)$ di tipo Stack Graph, e sia $r \in V$ il nodo che corrisponde alla reference da risolvere. Questo nodo sara' il nodo iniziale della ricerca: dopo aver inizializzato una coda $Q$ vuota, si costruisce il path $p_r = (r,)$ che ha come nodo di partenza il nodo iniziale $r$ e si inserisce il suddetto path $p$ nella coda $Q$.

\par
A questo punto inizia l'algoritmo vero e proprio, il quale altro non e' che un ciclo che termina quando la coda $Q$ e' vuota.
Ad ogni iterazione del ciclo si estrae un path $p$ dalla coda $Q$: si applica una funzione euristica $S : E \rightarrow bool$.
Questa funzione specifica se e' necessario, o piu' che altro conveniente, analizzare il path appena estratto, specialmente per evitare di rianalizzare due volte lo stesso path (o un path simile) e quindi liberarsi dai cicli infiniti.

TODO: spiega il cycle detector

\par
A questo punto, se il path deve essere analizzato, la prima cosa che occorre fare e' controllare se e' un path definito "completo", ovvero se il Symbol Stack e' vuoto e il nodo finale del path e' un nodo di tipo "definition": in questo caso la reference e' stata trovata e si puo' interrompere la ricerca.
Ulteriori informazioni di circostanza ("defkind" e "refkind" per esempio) possono definire regole ulteriori per definire quali definizioni devono essere accettate: nella soluzione proposta si va a valutare solo i path completi i cui nodi finali siano "definition" con un "defkind" non nullo e i cui nodi iniziali siano "reference" con "refkind" non nulla. Chiaramente la seconda condizione puo' essere controllata nel momento stesso in cui si decide di risolvere un nodo di reference.

\par
Se il path corrente non e' "completo", lo si va ad "estendere": per ogni arco del nodo finale si verifica la possibilita' di attraversarlo (in base al Symbol Stack e alle operazioni di pop) e in quel caso si aggiunge il nuovo path alla coda.

\par
E' bene notare come non solo ogni arco, ma anche ogni ciclo all'interno del grafo possono aggiungere molti path concorrenti alla coda.
Il ruolo della euristica in questo secondo caso e' evitare l'esplosione della coda: il tempo di computazione deve essere ragionevole (del resto l'obbiettivo e' appunto migliorare le prestazioni di costruzione del grafico delle dipendenze), e lo spazio usato dal programma deve essere limitato o non eccessivo.
Tuttavia in entrambi i casi le ottimizzazioni della grammatica TSG possono avere un impatto notevole sia sul numero di path concorrenti sia sulle prestazioni della risoluzione delle reference. E' quindi bene studiare bene ogni possibilita' che le feature di Stack Graphs offrono al programmatore.

\subsection{Uso di push e pop}

%TODO: quote
Generalmente i Query Language Tree Sitter dei linguaggi di programmazione introducono nodi atomici per gli identificatori: nel sono un esempio \emph{} e \emph{} per java.
Inoltre i costrutti che concatenano piu' identificatori, come per esempio \emph{qualified name} e dichiarazioni di tipo, sono rappresentati come nodi contenenti i nodi che compongono la stringa, che possono essere etichettati o meno.
Questo apre la possibilita' di risolvere in maniera ricorsiva Bottom Up ed incrementale la costruzione di definizioni e riferimenti: una volta costruiti i sottografi per i nodi sintattici figli si puo' riorganizzare la struttura finale del sottografo del nodo padre combinando i sottografi stessi e riassegnando variabili che identificano i punti di accesso o combinazione degli stessi.

\par
Si assegna sistematicamente ai nodi sintattici atomici alcuni nodi rispettivamente di "pop" e "push" per definizioni e riferimenti, quindi utilizzando archi e varabili si combinano questi nodi.
Infatti da una parte concatenando con un arco due nodi di "push" $X$ e $Y$ otterremo una ricerca che andra' a valutare prima l'esistenza di una definizione del simbolo associato a $Y$ e successivamente la definizione per il simbolo di $X$, e dall'altra l'esplorazione dall'alto potra considerare consecutivamente i due nodi $X$ e $Y$, per esempio per catturare relazioni di parentela o di contenimento.

\par
Questo meccanismo, a cui si fara' riferimento sotto il nome di \emph{resolution bridge}, e' molto utile per risolvere le reference nel codice che fanno uso di import o di inferenza di tipo o di classe: per esempio attraverso la concatenazione dei nodi di pop della dichiarazione di una variabile con i nodi di push del tipo dichiarato si costruisce incrementalmente la ricerca che dopo aver risolto il nome della variabile, se rimangono simboli sul Symbol Stack (per esempio se si cerca di catturare una relazione di \emph{field access}), potra' caricare sul Symbol Stack i simboli del tipo di dato e andare immediatamente alla ricerca del tipo da cui estrarre le informazioni rimanenti.

\par
Non tutti i nodi di reference costituiranno reference "concrete", ovvero che ha senso andare a risolvere: infatti se si ha una concatenazione di simboli si preferisce andare a ricercare solo la concatenazione invece di ogni singola sottocatena, visto che lo scopo dell'analisi e' ricercare le dipendenze tra componenti e unita' del codice. Si applica un filtro ancorando ai soli nodi finali di una catena di reference la variabile di debug "refkind", sia per distinguere il tipo di reference che per marcare quelle interessanti.

\par
Lo stesso vale in linea di principio per le definizioni: oltre alla risoluzione delle reference si applica anche un'esplorazione dall'alto per catturare i nodi del grafo delle dipendenze: package, classi, funzioni, campi ... etc. Sia per contrassegnare tra una definizione concreta, sia per distinguere tra definizioni di tipo diverso si usa la variabile "defkind".

\par
Durante la stesura di una concatenazione di nodi "push" che vengono usati per il \emph{resolution bridge}, per limitare il numero di path che vengono creati durante la ricerca, si puo' applicare un nodo di "pop" con lo stesso simbolo di un nodo di "push" concatenandolo prima di esso per permettere di usare questo ponte solo ai path che abbiano effettivamente sul Symbol Stack i simboli necessari per risolvere quella reference con quel \emph{resolution bridge}.
E' possibile farlo con ogni nodo "push" del ponte, tuttavia ci si limitera' al primo nodo di esso sia per semplicita' sia per tentare di limitare la dimensione del grafo, il quale gia' di per se avra' dimensioni notevoli.

\subsection{Struttura degli archi}

Il grafo delle dipendenze verra' creato tramite una esplorazione del grafo a partire dalla radice, quindi ogni nodo che contiene definizioni o reference utili deve essere raggiungibile da tale algoritmo.

\par
Non e' tuttavia l'unico vincolo che e' necessario imporre alla struttura degli archi: nell'ottica di ridurre il numero di path creati durante l'esplorazione e' nessario ridurre al minimo i cicli.
Questa ottimizzazione non e' complessa da adottare, ma il grafo non puo' essere completamente aciclico: deve essere possibile durante la ricerca ritornare quanto mento alla radice del grafo (\emph{ROOT\_NODE}) per permettere ad essa di discendere nel prossimo sottoalbero da esplorare per trovare le definizioni in successione.

\par
Per evitare cicli interni ogni nodi che definisce scope fara' progredire due nodi distinti che corrisponderanno rispettivamente al percorso verso l'alto e verso il basso: questo permettera' sia alle reference di fluire verso la radice, appendendo eventuali simboli che definiscono lo scope (ex: nome della classe), sia alle definizioni e alle reference stesse di essere identificate durante la fase di risoluzione delle reference e durante la fase di esplorazione dall'alto.

\section{Le Componenti di Base}

TODO: spiega l'interfaccia delle unita'

\subsection{Identificatori}

Per ogni nodo sintattico identificatore si creano sia i nodi push che i nodi pop per permettere alle altre unita' di incorporarlo in composizioni piu' complesse.

\putimage{diagrams/04/identifier.png}{"Identifier Unit"}{fig:identifier}

Un esempio di regola TSG per ottenere questo pattern e':

\begin{lstlisting}
(identifier)@this {
    node push
    node pop
    node scope
    let @this.push_start = push
    let @this.push_end = push
    let @this.pop_start = pop
    let @this.pop_end = pop
    let @this.scope = scope
    push -> scope
}
\end{lstlisting}

Questa struttura permette di comporre l'unita' per esempio nelle catene di identificatori, che generalmente in Tree Sitter possono essere formate in questo modo:

\begin{lstlisting}
    node scope
    let @this.push_start = @name.push_start
    let @this.push_end = @scope.push_end
    let @this.pop_start = @scope.pop_start
    let @this.pop_end = @name.pop_end
    let @this.scope = scope
    @scope.pop_end -> @name.pop_start
    @name.push_end -> @scope.push_start
    scope.push_end -> scope
\end{lstlisting}

Questo pattern e' ricorsivamente applicabile attraverso la cattura che usa Tree Sitter Graph, quindi scalabile in verticale: essa permettera' di compore unita' Scoped Identifier per un numero arbitrariamente grande di identificatori nella catena.

\putimage{diagrams/04/scopedIdentifier.png}{Scoped Unit}{fig:scopedIdentifier}

In questo modo si mantiene la medesima interfaccia per Scoped Identifier, con conseguente "polimorfismo" dell'argomento "scope" dello snippet TSG. Il risultato delle regole sara' quindi una catena di nodi che implementano la catena di identificatori.

\subsection{Classi, Interfacce, Enum e Annotazioni}
%TODO

Tutti questi costruitti hanno in comune il fatto di contenere propriamente tutte le sotto dichiarazioni al livello di codice sorgente, per cui si prendera' come esempio le classi o le interfacce in base alle relazioni introdotte.
Quindi avremo una struttura del genere a livello di Query Language Tree Sitter per il caso delle classi:

\begin{lstlisting}
  (class_declaration name: (_)@name body: (_)@body)@this
\end{lstlisting}

Esplorando l'unita' dall'alto si dovra' da una parte riconoscere la definizione di una classe e dall'altra poter definire che tutto il "contenuto" di una classe e' "contenuto".
In generale questo viene fatto nella grammatica TSG tramite l'ordine di esplorazione dei nodi.
La definizione e' introdotta tramite l'unita' identificatore \emph{name}, quindi si predispone un \emph{defkind} appropriato per quella definizione:

\begin{lstlisting}
  (class_declaration name: (_)@name)@this {
    attr (@name.pop_end) debug_defkind = "class"
  }
\end{lstlisting}

Quindi si deve vincolare l'esplorazione dell'unita' in modo che si percorrano in serie l'unita' \emph{name} e l'unita' \emph{body} dall'alto, e quindi vice versa dalle foglie.
Per fare questo si concatenano le due unita' fra loro e quindi con i rispettivi nodi di ingresso / uscita dell'unita' \emph{class}: 

\begin{lstlisting}
  (class_declaration name: (_)@name body: (_)@body)@this {
    edge @name.scope -> @this.scope
    edge @this.pop_start -> @name.pop_start
    let @this.pop_end = @name.pop_end
    edge @body.scope -> @this.scope
    edge @body.scope -> @name.push_start
    edge @name.pop_end -> @body.pop_start
  }
\end{lstlisting}

Si procede ora con l'ereditarieta'.
Sia nelle interfacce che nelle classi la struttura da applicare all'unita' e' pressocche' identica. In linguaggi che non distinguono tra classi e interfacce (come per esempio Python) l'implementazione e' identica, mentre in Java per esempio e' necessario introdurre due regole distinte. Questo perche' nei linguaggi dove si distingue tra i due costruitti sono implementate "operazioni" distinte per applicare super interfacce o super classi. E' il caso di Java con "implements" e "extends". Di conseguenza e' necessario supportare due relazioni di dipendenza distinte: "isImplementationOf" e "isChildOf".

L'Estensione e' un'operazione che riguarda sia le interfacce che le classi: entrambe possono essere figlie di un'altra interfaccia o classe, mentre l'Implementazione e' specifica delle classi.
Per permettere agli algoritmi di rilevare le due relazioni e poterle usare per risolvere l'ereditarieta' e quindi l'appartenenza di un campo o un metodo della super classe o super interfaccia si marca la reference con il \emph{refkind} appropriato e quindi si concatena l'unita' della classe con l'unita' \emph{super}:

\begin{lstlisting}
  [
    (class_declaration superclass: (superclass (_)@super))@this
    (interface_declaration (extends_interfaces (type_list (_)@super)))@this
  ] {
    attr (@super.push_start)
      is_reference, refkind = "extends"
    edge @this.pop_end -> @super.push_start
    edge @super.push_end -> @this.scope
  }
\end{lstlisting}

\begin{lstlisting}
  (class_declaration interfaces: (super_interfaces (type_list (_)@super)))@this {
    attr (@super.push_start)
      is_reference, refkind = "implements"
    edge @this.pop_end -> @super.push_start
    edge @super.push_end -> @this.scope
  }
\end{lstlisting}

Rimane solo l'applicazione dei modificatori, che nel caso di java sono le annotazioni: la procedura e' simile, ma la relazione in questo caso e' "usesType".

\begin{lstlisting}
  [
    (class_declaration (modifiers (marker_annotation name: (_)@type)) name: (_)@name)@this
    (class_declaration (modifiers (annotation name: (_)@type)) name: (_)@name)@this
  ] {
    attr (@type.push_start)
      is_reference, refkind = "usesType"
    edge @name.pop_end -> @type.push_start
    edge @type.push_end -> @this.scope
  }
\end{lstlisting}

\subsection{Package}

Nei linguaggi dove il nodo sintattico della definizione di package contiene i node delle definizioni e' possibile implementare un modello simile a quello impiegato per classi e interfacce.
In Java pero' la dichiarazione di package e' uno statement all'inizio del file, quindi bisognera' trovare un'altro modo per forzare l'ordine di esplorazione per far attraversare in serie lo statement e le definizioni nel file.

Prima di tutto si distinguono due casi: il file corrente puo' avere o non avere una dichiarazione di package.
Nel caso in cui non sia presente un package nel file si instaura un semplice collegamento tra il nodo \emph{program} (radice sintattica del file in Java) e il nodo dello Stack Graph ROOT\_NODE:

\begin{lstlisting}
  (program
    (package_declaration)? @package
  )@this {
    if none @package {
      edge ROOT_NODE -> @this.pop_start
      edge @this.scope -> ROOT_NODE
    }
  }
\end{lstlisting}

Nel secondo caso, prima si traslano le variabili dell'unita' \emph{name} del package all'interno del nodo sintattico della dichiarazione e si marca la dichiarazione attraverso il \emph{defkind} come "package":

\begin{lstlisting}
  (package_declaration
    (_)@name
  )@this {
    attr (@name.pop_end) defkind = "package"
    let @this.pop_start = @name.pop_start
    let @this.pop_end = @name.pop_end
    let @this.push_start = @name.push_start
    let @this.push_end = @name.push_end
  }
\end{lstlisting}

Quindi si infrappone l'esplorazione dell'unita' \emph{name} sia verso l'alto che verso il basso:

\begin{lstlisting}
  (program
    (package_declaration)? @package
  )@this {
    if none @package {
      edge ROOT_NODE -> @this.pop_start
      edge @this.scope -> ROOT_NODE
    } else {
      edge ROOT_NODE -> @package.pop_start
      edge @package.pop_end -> @this.pop_start
      edge @this.scope -> @package.push_start
      edge @package.push_end -> ROOT_NODE
      edge @this.scope -> ROOT_NODE
    }
  }
\end{lstlisting}

\subsection{Variabili}

Variabili (di metodo e di classe) e parametri dei metodi possono essere trattati allo stesso modo, ma nel modello puo' essere utile distinguere tra essi tramite il \emph{defkind}: rispettivamente "attribute" e "paramter".
Per via di questa similitudine di seguito si trattera' delle variabili di metodo.

Per ottenere la dichiarazione di una variabile si usa la solita regola TSG che assegna un defkind:

\begin{lstlisting}
  (local_variable_declaration
    declarator: (variable_declarator name: (_)@name)
  )@this {
    attr (@name.pop_end) defkind = "attribute"
  }
\end{lstlisting}

Quindi si collega e assegna la reference al tipo della variabile dichiarata:

\begin{lstlisting}
  (local_variable_declaration type: (_)@type declarator: (variable_declarator name: (_)@name))@this {
    attr (@type.push_start)
      is_reference, refkind = "usesType"
    edge @type.push_end -> @this.scope
    edge @name.pop_end -> @type.push_start
  }
\end{lstlisting}

\subsection{Metodi}

Metodi e funzioni possono essere trattati in maniera simile a classe e interfaccie, l'unica reale differenza e' la presenza del tipo di ritorno e dei parametri.
Se il tipo puo' essere utilizzato come nelle variabili, i parametri di per se costituiscono delle unita' autonome, esattamente come cio' che e' contenuto nei body, per cui l'unica azione particolare necessaria per questa unita' sara' collegare la lista dei parametri esattamente come viene collegato il body:

\begin{lstlisting}
  (method_declaration name: (_)@name parameters: (_)@parameters)@this {
    edge @parameters.scope -> @this.scope
    edge @name.pop_end -> @parameters.pop_start
  }
\end{lstlisting}

\subsection{Accesso ai Campi}
L'accesso ai campi puo' essere trattato in maniera molto simile alla concatenazione di identificatori: ricorsivamente si definisce una reference attraverso il \emph{refkind} e quindi si collega essa all'unita' \emph{object}:

\begin{lstlisting}
  (field_access object: (_)@object field: (_)@field)@this {
    edge @field.push_end -> @object.push_start
    let @this.push_start = @field.push_start
    let @this.push_end = @object.push_end
    edge @this.push_end -> @this.scope
    edge @this.pop_start -> @this.push_start
    attr (@field.push_start)
      is_reference, refkind = "accessField"
  }
\end{lstlisting}

Ci sono pero' altri tipi di accesso: per esempio l'accesso ad un elemento di un array o l'accesso al contenuto di una variabile nello scope. A seconda del linguaggio analizzato il Query Language potrebbe possedere costruitti appositi, come in java, o trattarli in maniera simile all'esempio precedente.
Nel secondo caso ogni reference particolare deve essere segnalata con un'apposita regola.
L'esempio seguente marca come reference da risolvere gli identificatori passati come argomento di una chiamata di funzione, utilizzati in espressioni binarie o che siano nomi di array:

\begin{lstlisting}
  [
    (binary_expression left: (identifier)@child)@this
    (assignment_expression left: (identifier)@child)@this
    (binary_expression right: (identifier)@child)@this
    (assignment_expression right: (identifier)@child)@this
    (array_access array: (identifier)@child)@this
    (argument_list (identifier)@child)@this
  ] {
    attr (@child.push_start)
      is_reference, refkind = "accessField"
    edge @this.pop_start -> @child.push_start
  }
\end{lstlisting}

\subsection{Chiamata di Metodi}

L'invocazione di metodi e' un altro esempio di relazione dove si deve distinguere il caso tra un identificatore ed una catena di identificatori. Nel caso di Java il Query Language mette a disposizione un costrutto unico per entrambi i casi:

\begin{lstlisting}
  (method_invocation object: (_)? @object name: (_)@name)@this
\end{lstlisting}

Nel caso piu' semplice si ha solo l'unita' \emph{name}, quindi si segnala la reference con il \emph{refkind} e si instaura un collegamento tra \emph{this} e \emph{name}:

\begin{lstlisting}
  (method_invocation object: (_)? @object name: (_)@name)@this {
    if none @object {
      attr (@name.push_start)
        is_reference, refkind = "calls"
      edge @this.push_start -> @name.push_start
      edge @name.push_end -> @this.scope
      edge @this.pop_start -> @name.push_start
      let @this.push_end = @name.push_end
    }
  }
\end{lstlisting}

Nel caso in cui il nome del metodo sia composto, l'unica differenza rispetto alla regola precedente e' l'inserimento in mezzo al collegamento dell'unita' \emph{object}, esattamente come e' stato fatto per il caso del package.

\subsection{Inclusione}
%TODO

Le unita' costruite dagli statement di inclusione dovranno sia segnalare la presenza di una relazione "includes", sia supportare la ricerca del modulo incluso di oggetti che non sono stati definiti nel modulo corrente.
Grazie all'unita' della catena di identificatori questo e' reso semplice e scalabile: si assegna il \emph{refkind} e si collega l'unita' \emph{name} al \emph{ROOT\_NODE} in modo che un path possa caricare i simboli dell'import e poi tornare alla radice per discendere nel modulo incluso.

\begin{lstlisting}
  (import_declaration (_)@name)@this {
    attr (@name.push_start) is_reference, refkind = "includes"
    let @this.pop_start = @name.pop_end
    edge @this.pop_start -> @name.push_start
    edge @name.push_end -> ROOT_NODE
  }
\end{lstlisting}

\subsection{File}

Esistono dei linguaggi dove il nome dei moduli e' uguale o strettamente legato al nome del file (come in Python), o dove l'inclusione e' effettuata specificando il nome del file da includere.
In questi casi e' necessario manipolare i nomi dei file per poter costruire un'unita' simile al package Java, che pero' presenta definizione con defkind distinta.
Questo permette di costruire relazioni di inclusione che utilizzano il percorso del file per identificare un certo modulo e poter raccogliere anche informazioni su quale file definisce i vari costrutti del linguaggio.
Di seguito un esempio con il linguaggio C++.

Prima di costruire l'unita' \emph{file} utilizzano il path relativo del file rispetto alla cartella di esplorazione del sorgete: questo e' fatto manualmente dal software che utilizza la grammatica TSG e Tree Sitter Stack Graph andando a leggere i file dopo essersi spostati nella cartella del progetto bersaglio. Questo permette di avere percorsi piu' concisi e meno dispersivi.

\begin{lstlisting}
(translation_unit)@this {
    let filepath = (path-normalize FILE_PATH)
    node @this.filepath
    attr (@this.filepath)
        symbol_definition = filepath, defkind = "file"
    edge ROOT_NODE -> @this.filepath
    edge @this.filepath -> @this.scope
    edge @this.scope -> ROOT_NODE
}
\end{lstlisting}

Quindi si risolve grazie al DSL di Tree Sitter Stack Graph il path del file incluso nel caso di percorso relativo e si assegna la relazione "includes":

\begin{lstlisting}
(preproc_include path: (string_literal)@path)@this {
    let mod_name = (path-normalize (path-join (path-dir FILE_PATH) (replace (source-text @path) "\"" "")))
    attr (@this.push_start)
        symbol_reference = mod_name, refkind = "includes"
    edge @this.push_start -> ROOT_NODE
    edge @this.pop_start -> @this.push_start
}
\end{lstlisting}

